% Created 2020-07-14 Tue 05:12
% Intended LaTeX compiler: pdflatex
\documentclass[11pt]{article}
\usepackage[utf8]{inputenc}
\usepackage[T1]{fontenc}
\usepackage{graphicx}
\usepackage{grffile}
\usepackage{longtable}
\usepackage{wrapfig}
\usepackage{rotating}
\usepackage[normalem]{ulem}
\usepackage{amsmath}
\usepackage{textcomp}
\usepackage{amssymb}
\usepackage{capt-of}
\usepackage{hyperref}
\author{Abhishiv Saxena}
\date{\today}
\title{Valuation of Firms}
\hypersetup{
 pdfauthor={Abhishiv Saxena},
 pdftitle={Valuation of Firms},
 pdfkeywords={},
 pdfsubject={},
 pdfcreator={Emacs 26.3 (Org mode 9.3.7)}, 
 pdflang={English}}
\begin{document}

\maketitle
\tableofcontents

\section*{{\bfseries\sffamily DONE} Restating Financial Statements}
\label{sec:org9d70abf}
\subsection*{Definitions}
\label{sec:orgd2f9d0c}
\subsubsection*{Accounting}
\label{sec:org2f4403e}
\begin{itemize}
\item ASSETS = LIABILITIES + EQUITY
\label{sec:org8e2fd7a}
\begin{itemize}
\item Operating Assets = Operating Liabilities + LT Debt + Equity
\label{sec:org538689d}
\end{itemize}
\item WORKING CAPITAL: The categories related to the cycle of business operations i.e. inventory, accounts receivable, and accounts payable.
\label{sec:org0c6671f}
\item 
\label{sec:orge89387a}
\end{itemize}
\subsubsection*{Corporate Finance}
\label{sec:org239a33c}
\begin{itemize}
\item RETURN = PROFIT/INVESTMENT
\label{sec:org89e9a3e}
\item FCF = NOPLAT − Net Increase in Invested Capital
\label{sec:org4035d59}
\end{itemize}
\subsubsection*{Basic Principles}
\label{sec:orgab6628b}
\begin{itemize}
\item Logic of differentiation between operating/financial parts of company during valuation
\label{sec:org6fec9e2}
\item Need for reorganization of financial statements
\label{sec:orgee3e863}
\end{itemize}
\subsection*{Calculating Investment(IC = "Investment Capital")}
\label{sec:org5caee77}
\subsubsection*{Theory}
\label{sec:org67677a5}
\begin{itemize}
\item Invested Capital = LT Debt + Equity = Operating Assets − Operating Liabilities
\label{sec:org6adcc58}
\item Total Funds Invested = Invested Capital + Non Operating Assets
\label{sec:orgfafbba5}
\end{itemize}

\subsection*{Calculating Return(NOPLAT = "Net Operating Profit Less Adjusted Taxes")}
\label{sec:org5e850ac}
\subsubsection*{Theory}
\label{sec:orgf34011c}
\begin{itemize}
\item NOPLAT is operating profit available to both debt/equity holders.
\label{sec:org9924a57}
\end{itemize}
\subsubsection*{Procedure}
\label{sec:orgc9aaace}
\begin{itemize}
\item First, interest is not subtracted from operating profit, because interest is considered a payment to the company’s financial investors, not an operating expense
\label{sec:org66acf2d}
\item Second, exclude any nonoperating income generated from assets that were excluded from invested capital
\label{sec:org0ddee74}
\item Thirdly, to calculate operating taxes, start with reported taxes, add back the tax shield caused by interest expense, and remove the taxes paid on nonoperating income.
\label{sec:orgadfe9e8}
\end{itemize}
\subsubsection*{Calculations}
\label{sec:org3200477}
\begin{itemize}
\item Original Financial Statememnts
\label{sec:orgc4cfee8}
\begin{itemize}
\item Sales(CampariOperatingSalesStatement)
\label{sec:org4d211e1}
\begin{table}[htbp]
\label{CampariOperatingSalesStatement}
\centering
\begin{tabular}{lrr}
\textbf{Sales Statement} & 2015 & 2016\\
\hline
\textbf{Total sales} & 1657 & 1727\\
 &  & \\
\hline
\end{tabular}
\end{table}
\item Cost(CampariOperatingCostStatement)
\label{sec:org05ea3dd}
\begin{table}[htbp]
\label{CampariOperatingCostStatement}
\centering
\begin{tabular}{lrr}
\textbf{Cost Statement} & 2015 & 2016\\
\hline
Cost of goods sold (COGS) & -740 & -742\\
Advertising and promotion & -286 & -309\\
SGAs & -298 & -323\\
One-off items & -23 & -33\\
\hline
\textbf{Total Cost} & -1347 & -1407\\
\hline
\end{tabular}
\end{table}

\item Non Operating Income(CampariNonOperatingRevenueCostStatement)
\label{sec:org65e238d}
\begin{table}[htbp]
\label{CampariNonOperatingRevenueCostStatement}
\centering
\begin{tabular}{lrr}
\textbf{Non Operating Revenue and Cost} & 2015 & 2016\\
\hline
Financial income & 8 & 15\\
Financial expenses & -69 & -74\\
Put options costs & 0 & 1\\
\hline
One-off's finacial expenses & 1 & -25\\
Income from associates & 0 & 0\\
\hline
\textbf{Total Non Operating Income} & -60 & -83\\
\hline
\textbf{Total Interest Income/Expense} & -61 & -58\\
\hline
\textbf{Total Non-Interest Income/Expense} & 1 & -25\\
\hline
\end{tabular}
\end{table}

\item Income Statement(CampariIncomeStatement)
\label{sec:orgd92a075}
\begin{table}[htbp]
\label{CampariIncomeStatement}
\centering
\begin{tabular}{lrr}
\textbf{Income Statement} & 2015 & 2016\\
\hline
\textbf{Net sales} & 1657 & 1727\\
\hline
\textbf{Cost} & -1347 & -1407\\
\hline
\textbf{Operating Profit} & 310 & 320\\
\hline
\textbf{Other Income} & -60 & -83\\
\hline
\textbf{Pre-tax profits} & 250 & 237\\
\hline
\textbf{Taxes} & -73 & -71\\
\hline
\textbf{Net profit} & 177 & 166\\
\hline
\end{tabular}
\end{table}

\item D\&A Opex
\label{sec:orgeb5164f}
\begin{table}[htbp]
\label{CampariDAOpex}
\centering
\begin{tabular}{lrr}
\textbf{D\&A included into OPEX} &  & \\
\hline
COGS & -33.4 & -37.2\\
Advertising and promotion & -0.7 & -0.7\\
SGAs & -13.2 & -14.8\\
\hline
\textbf{Total D\&A} & -47.3 & -52.7\\
\end{tabular}
\end{table}
\end{itemize}

\item Restated Financial Statements
\label{sec:org8bf39d9}
\begin{itemize}
\item Cost(CampariRestatedOperatingCostStatement)
\label{sec:org6ddddf1}
Important: \textbf{subtract} D\&A \textbf{expense}, so as to add it back.\\
\begin{table}[htbp]
\label{CampariRestatedOperatingCostStatement}
\centering
\begin{tabular}{lrr}
\textbf{Restated Cost Statement} & 2015 & 2016\\
\hline
Cost of goods sold (COGS) & -706.6 & -704.8\\
Advertising and promotion & -285.3 & -308.3\\
SGAs & -284.8 & -308.2\\
\hline
\textbf{Total Cost} & -1276.7 & -1321.3\\
\end{tabular}
\end{table}

\item Income Statement(CampariRestatedIncomeStatement)
\label{sec:org09e3184}
\begin{table}[htbp]
\label{CampariRestatedIncomeStatement}
\centering
\begin{tabular}{lrr}
\textbf{Income Statement} & 2015 & 2016\\
\hline
\textbf{Net sales} & 1657 & 1727\\
\hline
\textbf{Cost} & -1276.7 & -1321.3\\
\hline
\textbf{EBITDA} & 380.3 & 405.7\\
\hline
\textbf{D\&A} & -47.3 & -52.7\\
\hline
\textbf{EBIT} & 333. & 353.\\
\hline
\textbf{Net Interest} & -61 & -58\\
\hline
\textbf{Other Income} & 1 & -25\\
\hline
\textbf{OneoffItems Income} & -23 & -33\\
\hline
\textbf{EBT} & 250. & 237.\\
\hline
\end{tabular}
\end{table}
\end{itemize}
\end{itemize}

\subsection*{Calculating Cashflow(FCF = "Free Cash Flow")}
\label{sec:org9cfadea}

\section*{{\bfseries\sffamily DONE} Estimating Cost of Capital}
\label{sec:orgbb5f683}
\subsection*{Definitions}
\label{sec:org7b4b052}
\subsubsection*{K\textsubscript{eu}: unlevered cost of equity}
\label{sec:orga2225c2}
Rate of return acceptable by equity investors in a completely unlevered firm. Reflects the risk profile associated with corporate assets ignoring capital structure.\\
\subsubsection*{K\textsubscript{d}: levered cost of equity}
\label{sec:org389c440}
Rate of return acceptable by a form debt.\\
\subsubsection*{WACC = (D/V)*K\textsubscript{d}*(1 - T) + (E/V)*k\textsubscript{e}}
\label{sec:orgfea7c0a}
\subsection*{Estimated unlevered cost of equity}
\label{sec:org3d5e582}
\subsubsection*{CAPM}
\label{sec:org4723a4c}
R\textsubscript{i} = r\textsubscript{f} + B\textsubscript{i}[R\textsubscript{m} - r\textsubscript{f}]\\
Here,  Beta represents a stock’s incremental risk to a diversified investor, where risk is defined as the extent to which the stock covaries with the aggregate stock market. High beta means company must earn and offer higher return to entice investors.\\
\begin{itemize}
\item Estimating Risk free rate r\textsubscript{f}: use zero coupon treasury bonds
\label{sec:org8d2b7a7}
\item Estimating Market risk premium: 5-6\%
\label{sec:org008d323}
\item Estimating Beta
\label{sec:org2c23043}
Use industry beta, instead of company beta. To undo/normalise the effect of leverage use Modiligani and Miller.\\
\end{itemize}
\subsection*{Estimating cost of debt}
\label{sec:org1e07710}
After-Tax Cost of Debt = Cost of Debt × (1 − T\textsubscript{m})\\
\section*{{\bfseries\sffamily TODO} Multiples}
\label{sec:org5c7aca4}
The multiples method seeks to develop a relationship between the actual price of shares of comparable listed companies and an accounting metric (such as net income, cash flows, revenues, etc.)\\

\begin{center}
\begin{tabular}{ll}
\hline
Enterprise Value & Equity Value\\
\hline
EV/EBIT & Price by Earning: P/E\\
EV/EBITDA & Price by Cash earning: P/CE\\
EV/Sales & Price by Book Value: P/BV\\
\hline
\end{tabular}
\end{center}

\subsection*{Theory}
\label{sec:orge957a14}
\subsubsection*{Assumptions}
\label{sec:orgac09111}
\begin{itemize}
\item The company’s value changes proportionally to the internal variables chosen as a measure of performance;
\label{sec:org31f75ad}
\item The growth rates of both cash flows and the risk level are constant
\label{sec:org96036fb}
\end{itemize}
\subsubsection*{Concept}
\label{sec:orgf9897e9}
\begin{itemize}
\item The financial method creates cash flow projections based on estimations on cost of capital, and projection on growth rate.\\
\item The multiples method avoids these estimations and takes the expected growth rate and risk appreciation directly from market data through the use of multiples.\\
\end{itemize}
\begin{itemize}
\item Stock and Deal Multiples
\label{sec:org106cc81}
Multiples can refer to values taken from two different market contexts:\\
\begin{itemize}
\item The stock market\\
\item The market for corporate control\\
\end{itemize}
MARKET VALUE = STAND ALONE VALUE + PREMIUM\\
\item Financial and Business Multiples
\label{sec:org35d1840}
\begin{itemize}
\item Financial multiples identify the connection between a equity market price and some important measures such as cash flows, net income, or EBITDA
\label{sec:orgbd90959}
\item Business multiples link the equity market price to specific elements which are relevant for the business model typical of the sector the company(RPU for telecom i.e.)
\label{sec:org13d5c7f}
\end{itemize}
\end{itemize}
\subsubsection*{Common multiples}
\label{sec:org2771b02}
\begin{itemize}
\item P or EV
\label{sec:orgdb7a484}
P can be calculated by EV by subtracting market value of debt.\\
\item Direct or Indirect
\label{sec:org46cfb52}
\begin{itemize}
\item Direct: Measure how much value company is generating P/E, or EV/EBITDA\\
\item Indirect: Unable to measure this - P/BV, or EV/Sales\\
\end{itemize}
\item Current, Trailing, and Leading Multiples
\label{sec:org7af4b03}
\begin{itemize}
\item Current: last available balance sheet\\
\item Trailing: last 12m\\
\item Leading: future results\\
\end{itemize}
\begin{center}
\begin{tabular}{lll}
\hline
Current & Trailing & Leading\\
P\textsubscript{0}/E\textsubscript{t0} & P\textsubscript{0}/E\textsubscript{ltm} & P\textsubscript{0}/E\textsubscript{t1}\\
\end{tabular}
\end{center}

We should be consistet in which time period to choose in calculated, and then applying that multiple to variables.\\
\end{itemize}

\subsection*{P/E}
\label{sec:orgc94a8ed}
\subsubsection*{Steady state}
\label{sec:org662c60a}
\subsubsection*{Growth}
\label{sec:orgc025481}
\subsection*{EV/EBIT \& EV/EBITDA}
\label{sec:orgf8576ed}
\subsubsection*{Steady state}
\label{sec:org5a622e8}
\subsubsection*{Growth}
\label{sec:orgd54d92a}
\subsection*{Other Multiples}
\label{sec:org9fa0619}
\subsection*{Leverage}
\label{sec:orgb314d3f}
\subsection*{Unlevered Multiples}
\label{sec:org28e8def}
\subsection*{Multiples and Growth}
\label{sec:orgc6ab68d}
\subsection*{PEG Ratio}
\label{sec:org235ab97}
\subsection*{Value Maps}
\label{sec:orgf22d5f3}
\section*{{\bfseries\sffamily DONE} Acquisition Value}
\label{sec:org818d7ca}
\subsection*{Definitions}
\label{sec:org08ba519}
\subsubsection*{Acquisition Value = Stand-alone Value + Merger synergies from the point of view of a buyer\hfill{}\textsc{Definition}}
\label{sec:org69e686c}
\subsubsection*{Fair Market Value = based on operation company + Merger synergies in group of buyers\hfill{}\textsc{Definition}}
\label{sec:orgaa5b578}
\subsubsection*{Three components of Acquisition Value}
\label{sec:orgb655d73}
\begin{itemize}
\item Stand-alone Value
\label{sec:org6217d7b}
\item Value from Incremental cash flows resulting for merger synergies in target firm.
\label{sec:org055fbec}
\item Value of merger synergies obtained in other businesses led by controlling shareholders.
\label{sec:org1926537}
\end{itemize}
\subsection*{Value Created by an Acquisition}
\label{sec:org7d89bcb}
NPV\textsubscript{acq} = W\textsubscript{acq} - P\\
or the value created by and acquisition is equal to the PV of cashflows of acquired firm less the price paid for it.\\
\subsubsection*{Differential approach}
\label{sec:org4fe81d5}
 A scheme of analysis comparing two scenarios 1) Base case: without any acquisition, and 2) Innovation case: post acquisition.\\
Therefore, value of acquisition of the target company = W\textsubscript{acq(B)} = W\textsubscript{A+B} - W\textsubscript{A}\\
\begin{itemize}
\item Effect on cash flow
\label{sec:org28ab19d}
\begin{itemize}
\item Incremental flows due to collusive policies and consolidation\\
\item Incremental flows due to gains in operational efficiency\\
\item Incremental flows due to gains of intangible natures(marketing etc)\\
\end{itemize}
\item Effect on risk profile
\label{sec:org248a5a4}
\begin{itemize}
\item Positive effects: Leverage over suppliers/buyers\\
\item Negative effects: in case of vertical integration\\
\end{itemize}
\item Effect on credit profile
\label{sec:org1dd0766}
\begin{itemize}
\item If debts increases, so does EV due to tax shields.\\
\end{itemize}
\end{itemize}
\subsubsection*{Benefits}
\label{sec:org69c5b6f}
W\textsubscript{acq} = W\textsubscript{base} + W\textsubscript{s} + W\textsubscript{o} where\\
\begin{itemize}
\item W\textsubscript{base} = base value\\
\item W\textsubscript{s} = value of synergies\\
\item W\textsubscript{o} = value of opportunities created by the acquisition\\
\end{itemize}
\subsection*{Value-Components Model}
\label{sec:orgca594a9}
\subsubsection*{Model}
\label{sec:org04c93f7}
\subsubsection*{Application}
\label{sec:orga3e0fe4}
\subsection*{Further Considerations in Valuing Acquisitions}
\label{sec:org987b802}
\subsubsection*{{\bfseries\sffamily DONE} Analysis Standpoint}
\label{sec:orgfe79708}
\subsubsection*{{\bfseries\sffamily TODO} Estimation of Cash flow}
\label{sec:org997b100}
\subsubsection*{{\bfseries\sffamily TODO} Time Horizon}
\label{sec:org2443501}
\subsubsection*{{\bfseries\sffamily TODO} Debit profile}
\label{sec:orgd24e6cd}
\subsubsection*{{\bfseries\sffamily TODO} Asset-Side versus Equity-Side}
\label{sec:org5af6987}
\subsection*{Acquisition Value of Plastic Materials Company}
\label{sec:orgb824bf4}
\subsubsection*{Assumption}
\label{sec:org426f5f2}
\subsubsection*{Synergy Estimation}
\label{sec:org83d4158}
\subsubsection*{Calculation}
\label{sec:org5081d1c}
\subsection*{Acquisition Value of Controlling Interests}
\label{sec:orge199f1a}
Usually, the value of a controlling stake is higher than the value of shares they represent.\\
W = W\textsubscript{Base Value} + W\textsubscript{diff.flows}\\
Where, W\textsubscript{diff.flows} can be divided into value from cash flows that are\\
\begin{itemize}
\item Divisible flows: Accuruing to target company and threfore available to all shareholders.\\
\item Indivisible flows: Accuring only to controlling interest or companies controlled by them\\
\end{itemize}
Therefore the value for block of x shares is:\\
\textbf{W\textsubscript{x} = (W\textsubscript{base} + W\textsubscript{div})*x + W\textsubscript{indiv}}\\
\subsubsection*{Private Benefits}
\label{sec:orgf35c380}
\subsection*{Other Determinants of Control Premium}
\label{sec:org148cf15}
\subsection*{Acquisition Value in a Mandatory Tender Offer}
\label{sec:orgedfc660}
The weighted price paid for 100 percent of the capital of the target company must be less than or equal to the total acquisition value of the shares.\\
P\textsubscript{pc}*a + [(P\textsubscript{pc} \_ P\textsubscript{m})/2]*(1 - a) < w\textsubscript{acq}\\
where\\
P\textsubscript{pc} = price per share bought off market\\
a = percentage of capital represented by above block\\
P\textsubscript{m} = avg market price\\
w\textsubscript{acq} = unit acquisition value\\
\subsection*{Maximum and Minimum Exchange Ratios in Mergers}
\label{sec:org4d20bcf}
\begin{itemize}
\item Maximum value of ER is one that has neutral impact on shares of \textbf{buying company}\\
\item Minimum value of ER is one that has neutral impact on shares of \textbf{target company}\\
\end{itemize}
\subsection*{Exchange Ratio and Third-Party Protection}
\label{sec:org6d28412}
\section*{{\bfseries\sffamily DONE} Value in Corporate Control}
\label{sec:org7d62a1b}
\subsection*{Price Formation}
\label{sec:orge5129be}
\subsubsection*{Competitive Market}
\label{sec:org9981b90}
The market for acquisitions is not competitive if, on the other hand, a significant number of firms is not interested in external growth opportunities.\\
\subsubsection*{Mechanism of Price formation}
\label{sec:org0a3fb59}
From microeconomic theory, at equilibrium estimate of value for target firm should be identical.\\
P = W\textsubscript{acq}\\
However, realistically W\textsubscript{acq} is probably different for each firm due to difference in market plan/strategy.\\
W\textsubscript{acq}y < P < W\textsubscript{acq}xi\\
\subsection*{Benefits arising from acquisitions}
\label{sec:orgde582bc}
\subsection*{Estimate adjusting stand-alone cash flows}
\label{sec:orgda60150}
\subsubsection*{Cutting fixed costs}
\label{sec:orgf1f81e7}
\subsubsection*{Estimating breakup costs}
\label{sec:orgec66be7}
Break-up value = Company's assets - Charge for realizing breakup\\
\subsubsection*{Endemic: Exploiting commercial network}
\label{sec:org7eb40dd}
\subsection*{Premiums and Discounts}
\label{sec:orgc300400}
\subsection*{Most common Premiums and Discounts}
\label{sec:org87a8a75}

\section*{{\bfseries\sffamily DONE} Rights Issue}
\label{sec:org9bbd651}
\subsection*{Definitions}
\label{sec:org8eab72b}
\begin{itemize}
\item Cum-Right price = p = Stock price just before announcement\\
\item Subscription Offer = Preemptive Rights granted to existing shareholders to buy stock  in order to raise  money :Definition:\\
\item Subscription Price = P\\
\item TERP = Theoretical Ex-Right Price\\
\item Discount to TERP = D = (TERP - P) / TERP\\
\item Gross discount = (p - P) / P\\
\item VR = Theoretical value of one right\\
\item R = Number of rights needed to subscribe to one share\\
\end{itemize}
\subsection*{Setting subscription price}
\label{sec:orga8f0e42}
From law of conversation of value\\
TERP = [(n * p) + (N * P) ] / (n + N)\\
\subsubsection*{Setting discount}
\label{sec:org156263a}
0 < D < (n * p) / (n*p + S)\\
\subsection*{Valuing Preemptive rights}
\label{sec:orgea2f912}
TERP = R * VR + P\\
\subsubsection*{Ex Right Price}
\label{sec:org4a63780}
During subscription period following relationship should hold\\
ERP = R * VR + P\\
\section*{{\bfseries\sffamily DONE} Carbon}
\label{sec:org316f386}
\subsection*{Definitions}
\label{sec:org1b7e67a}
\subsection*{Carbon Risk}
\label{sec:org5c59b96}
\subsubsection*{\textbf{Policy and Legal}: Policies or regulations that could impact the operational and financial viability of carbon assets i.e. fuel efficiency standards}
\label{sec:orgf1b590c}
\subsubsection*{\textbf{Technology}: Developments in the commercial availability and cost of alternative and low-carbon technologies i.e. energy storage, alternative fuels etc.}
\label{sec:orgfbf6bd5}
\subsubsection*{\textbf{Market and Economic}: Changes in market or economic conditions that would negatively impact carbon assets i.e. changes in fossil fuel prices or consumer preferences}
\label{sec:orgeb557e7}
\subsection*{Carbon Pricing}
\label{sec:orgc8c6df1}
\subsubsection*{Approaches}
\label{sec:org4d33ab4}
The objective of carbon pricing is to shift the social costs of damage back to those who are responsible for them, and who can actually curb them.\\
\begin{itemize}
\item Carbon taxes
\label{sec:org4ca20db}
\item Emission trading systems(ETS)
\label{sec:orgd628626}
\end{itemize}
\subsubsection*{External/Internal Carbon prices}
\label{sec:org651f4a1}
For managing internal projects.\\
\subsection*{Valuation with Carbon risk}
\label{sec:orgcc265fb}
\subsubsection*{Scenario-Based Valuation and Carbon}
\label{sec:org7f15ce2}
Pick number of scenarios and drivers. Value them, and create probability for each.\\
\begin{center}
\begin{tabular}{lrrr}
\hline
g/C\$ & \$15 & 34\$ & 60\$\\
\hline
4.5\% & 57 & 37 & 10\\
3\% & 56 & 36 & 10\\
1.5\% & 55 & 35 & 36\\
\hline
\end{tabular}
\end{center}

\subsubsection*{Stochastic Simulation Valuation and Carbon}
\label{sec:org6f2b4bd}
Monte Carlo - Crystal Ball\\
\subsection*{Carbon Beta}
\label{sec:org160605c}
\subsubsection*{Theory}
\label{sec:org07229b4}
Alternative to CAPM, a multi factor model which's more appropriate for companies exposed to carbon risks.\\
𝜇\textsubscript{i} =r\textsubscript{f} + 𝛽\textsubscript{iM}*(𝜇\textsubscript{M} - r\textsubscript{f}) + 𝛽\textsubscript{iCO2}*(𝜇\textsubscript{CO2} − r\textsubscript{f})\\
\subsubsection*{Application}
\label{sec:org2263f26}

\section*{Exercise 1}
\label{sec:orgb98afc1}
\subsection*{Problem 2\hfill{}\textsc{Beta}}
\label{sec:orgee2fba1}
ACME corporation is a utility company whose corporate tax rate is 33\%. The reference free risk rate is equal to 3\% and the expected market return is 9\%. Using data on beta comparable and assuming k\textsubscript{d} = k\textsubscript{ts} and β\textsubscript{d} = 0 shown in the following table determine the kEU value of the company.\\

\begin{table}[htbp]
\label{Ex1P2IndustryBetaTable}
\centering
\begin{tabular}{rrrrrrr}
\# & Accounts Payable & Mkt. Capital & Short term D & Long term D & Cash & Beta\\
\hline
1 & 876.0 & 16803.5 & 4564.0 & 0 & 1657.6 & 0.910\\
2 & 1478.1 & 17396.4 & 4376.4 & 2527.0 & 714.0 & 1.320\\
3 & 554.3 & 16790.2 & 6314.0 & 9695.0 & 213.5 & 1.500\\
\end{tabular}
\end{table}
\begin{table}[htbp]
\label{Ex1P2IndustryBetaCalculation}
\centering
\begin{tabular}{rrrr}
\# & Net debt & leverage ratio & Unlevered Beta\\
\hline
1 & 2906.4 & 17.296397 & 0.81549559\\
2 & 6189.4 & 35.578625 & 1.0659115\\
0 & 15795.5 & 94.075711 & 0.92007196\\
\hline
 &  &  & 0.93382635\\
\end{tabular}
\end{table}

Therefore K\textsubscript{eu} = r\textsubscript{f} + ß*MRP = 0.03 + 0.93382635*0.06\\

\subsection*{Problem 3\hfill{}\textsc{EV}}
\label{sec:org7e15311}
Consider the following data of Beta at year 2011 and year 2012 together with the data of the previous exercise and assuming a steady state scenario, calculate:\\
\begin{itemize}
\item Beta‚ asset unlevered EV, levered EV and equity value with APV (assuming risk free debt and k\textsubscript{txa} = k\textsubscript{d})\\
\item Beta‚ asset enterprise and equity value with DCF (assuming a D/E target ratio equal to 11\% and kD equal to 3\%)\\
\end{itemize}

\begin{center}
\begin{tabular}{lrr}
Company Beta & 2011 & 2012\\
\hline
EBIT & 4800.0 & 5340.0\\
Investment & 290.0 & 530.0\\
Divestiture & 0 & 340.0\\
Account payable & 410.0 & 380.0\\
Account receivable & 180.0 & 250.0\\
Inventories & 75.0 & 80.0\\
Excess cash and equivalents & 32.0 & 92.0\\
Short term Financial debt & 2560.0 & 2400.0\\
Long term Financial debt & 1900.0 & 1780.0\\
Depreciation & 140.0 & 150.0\\
\end{tabular}
\end{center}

\section*{Exercise 2}
\label{sec:orgba1c946}
\subsection*{Problem 1}
\label{sec:org45d6ace}
On the basis of the information provided, and applying a tax rate = 30\%, estimate:\\
\begin{itemize}
\item Industry average unlevered beta from the panel of comparable below and assuming kD = kTS and βD = 0\\
\item Alpha’s enterprise value in 2012 using the Adjusted Present Value (APV) model and assuming rf =3.5\% MRP = 5.0\%, kTS = 3.5\% and a growth rate g = 1.0\% for the years after the explicit forecast period\\
\end{itemize}

\begin{center}
\begin{tabular}{lrlrrrrl}
Comparables & Levered beta & Conv. bonds & ST debt & OL & cash & Market cap & Tax \%\\
\hline
Company A & 1.0 & - & 60.0 & 40.0 & 20.0 & 160.0 & 30.0\%\\
Company B & 1.2 & 800.0 & 80.0 & 200.0 & 100.0 & 600.0 & 25.0\%\\
Company C & 0.8 & - & 100.0 & 160.0 & - & 400.0 & 30.0\%\\
\end{tabular}
\end{center}


\begin{center}
\begin{tabular}{lrrr}
 & 2013 & 2014 & 2015\\
\hline
Free cash flows & 200.0 & 210.0 & 218.0\\
Tax shields & 20.0 & 22.0 & 24.0\\
\end{tabular}
\end{center}


\begin{center}
\begin{tabular}{ll}
Tax rate & 30.0\%\\
\hline
rF & 3.5\%\\
MRP & 5.0\%\\
kTS = kD & 3.5\%\\
Growth after the explicit forecast period & 1.0\%\\
Valuation date & 2012\\
Explicit period & 2013 - 2015\\
\end{tabular}
\end{center}

\subsection*{Problem 2}
\label{sec:org5981bb6}
Company Delta is listed on the market with 200 shares and a current price of 5. Delta needs to invest 500 in new projects and wants to raise funds on the market, but local regulation requires to give preemptive rights to existing shareholders. Delta issues a press release saying "raising 500 @ 20\% TERP discount". Estimate TERP, subscription price, new shares issued, gross discount and the trading value of the rights.\\
Re-work the previous exercise but starting with this alternative press release: "raising a total of 500, shares @ 3.5 each" (i.e. not knowing that TERP discount is 20\%).\\

\subsection*{Problem 3}
\label{sec:org23421ad}
Company A acquired and merged Company B. The acquisition was paid by issuing new stocks. The exchange ratio defined between parties was 0.55. Based on data provided below and assuming synergies equal to 400 determine the minimum and maximum exchange ratios for the parties and the return for shareholders of company A and company B.\\
\end{document}
